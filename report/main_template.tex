% IEEE Double-Column Template with Appendix
% Compile: pdflatex → bibtex → pdflatex → pdflatex

\documentclass[conference]{IEEEtran}

\IEEEoverridecommandlockouts

\usepackage[numbers]{natbib}
\usepackage{amsmath,amssymb,amsfonts}
\usepackage{algorithm}
\usepackage{algorithmic}
\usepackage{graphicx}
\usepackage{textcomp}
\usepackage{xcolor}
\usepackage{url}
\usepackage{booktabs}
\usepackage{multirow}
\usepackage{listings}

% Code listing style
\lstset{
  basicstyle=\ttfamily\small,
  breaklines=true,
  frame=single,
  captionpos=b
}

\begin{document}

% ---------- TITLE ----------
\title{A Intro to AI/ML Project Title in IEEE Double-Column Format}

\author{
\IEEEauthorblockN{Team Members Name seperated by Comma}
\IEEEauthorblockA{
Team Name \\}
}

\maketitle

% ---------- ABSTRACT ----------
\begin{abstract}
This document presents a data-mining research project following IEEE double-column formatting. The abstract summarizes the purpose, dataset, methodology, key results, and contributions of the work.
\end{abstract}

\begin{IEEEkeywords}
Data mining, machine learning, classification, feature engineering, evaluation.
\end{IEEEkeywords}

% ---------- INTRODUCTION ----------
\section{Introduction}
Introduce the domain, the motivation for the task, and why the problem matters. Clearly state the goal of the project.

% ---------- RELATED WORK ----------
\section{Related Work}
Discuss prior methods and related research in data mining.

% ---------- METHODOLOGY ----------
\section{Methodology}

\subsection{Dataset Description}
Describe dataset size, features, target, and characteristics.

\subsection{Preprocessing}
Explain normalization, imputation, encoding, etc.

\subsection{Feature Engineering}
Outline new features or selection methods.

\subsection{Models Used}
Describe the models evaluated.

\subsection{Training and Validation}
Explain metrics and validation strategy.

% ---------- RESULTS ----------
\section{Results}

\subsection{Quantitative Results}
\begin{table}[ht]
\centering
\caption{Model Performance Comparison}
\begin{tabular}{lcccc}
\toprule
Model & Acc & Prec & Rec & F1 \\
\midrule
LR & 0.78 & 0.74 & 0.70 & 0.72 \\
RF & 0.83 & 0.80 & 0.78 & 0.79 \\
XGB & \textbf{0.86} & \textbf{0.84} & \textbf{0.81} & \textbf{0.82} \\
\bottomrule
\end{tabular}
\end{table}

\subsection{Figures}
\begin{figure}[ht]
\centering
\includegraphics[width=0.45\textwidth]{example-figure.pdf}
\caption{Example figure placeholder.}
\label{fig:example}
\end{figure}

% ---------- DISCUSSION ----------
\section{Discussion}
Interpret results and highlight limitations.

% ---------- CONCLUSION ----------
\section{Conclusion}
Summarize major contributions and findings.

% ---------- REFERENCES ----------
\bibliographystyle{IEEEtranN}
\bibliography{references}

% ---------- APPENDIX ----------
\appendix

\section{GitHub Repository Information}

This appendix provides details regarding the project’s GitHub repository and associated file structure for reproducibility and open-source collaboration.

\subsection{Repository Link}
The complete source code, datasets (if permitted), trained models, and documentation are available at:

\begin{center}
\textbf{\url{https://github.com/YourUsername/YourRepositoryName}}
\end{center}

\subsection{Repository Structure}

The project repository follows the structure shown below:


\subsection{Description of Key Components}

\begin{itemize}
    \item \textbf{data/}: Contains raw, processed, and external datasets.
    \item \textbf{notebooks/}: Jupyter Notebooks for EDA, preprocessing, model training, and evaluation.
    \item \textbf{src/}: Modular Python source code for reproducible experiments.
    \item \textbf{models/}: Trained models and intermediate outputs.
    \item \textbf{results/}: Final plots, tables, and exported evaluation artifacts.
    \item \textbf{requirements.txt}: Exact Python dependencies.
    \item \textbf{README.md}: Instructions for running and reproducing results.
\end{itemize}

\end{document}
